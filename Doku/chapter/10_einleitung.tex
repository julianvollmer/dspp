%!TEX root = ../greenIT.tex
\section{Hintergrund}
Motivation dieser Arbeit ist das errichtete Forschungs- und Weiterbildungszentrums für Kultur und Informatik am Campus Wilhelminenhof der Hochschule für Technil und Wirtschaft (HTW Berlin). Das Gebäude wurde offiziell im März eröffnet. Besondere Merkmale an dem dem Gebäude sind die speziellen Fassaden. Die am wenigstens belichtete Nordseite des Gebäudes ist mit einer Medienfassade ausgestattet. Diese Fassade verfügt über drei unterschiedliche Visualisierungstechnologien, LED-Kacheln, Rückprojektion, und Ambient-Light. Ost-, Süd- und Westseite sind mit einer Photovoltaik-Anlage versehen, diese wird zur Energiegewinnung genutzt. Weiterhin ist das gesamte Gebäude mit einer ein Schüco Energiemanager ausgestattet. Über dieses Managementsystem können der aktuelle Energieverbrauch des Hauses abgelesen werden.

\section{Zielstellung}
Ziel dieser Arbeit ist die Bilanzierung der Energiegewinnung des Hauses. Es sollen Möglichkeiten aufgezeigt werden, mit denen es möglich das Gebäude möglichst effizient zu betreiben.